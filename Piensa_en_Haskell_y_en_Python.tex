% Piensa_en_Haskell_y_en_Python.tex
% Piensa en Haskell y en Python.
% José A. Alonso Jiménez <https://jaalonso.github.io>
% Sevilla, 15-diciembre-2022 (a 4-marzo-2023)
% ======================================================================

\documentclass[a4paper,12pt,twoside]{book}

%%%%%%%%%%%%%%%%%%%%%%%%%%%%%%%%%%%%%%%%%%%%%%%%%%%%%%%%%%%%%%%%%%%%%%%%
%% § Paquetes adicionales
%%%%%%%%%%%%%%%%%%%%%%%%%%%%%%%%%%%%%%%%%%%%%%%%%%%%%%%%%%%%%%%%%%%%%%%%

% Configuración para XeLaTeX
\usepackage{tocloft}
\setlength{\cftbeforechapskip}{2ex}
\setlength{\cftbeforesecskip}{0.5ex}
\setlength{\cftsecnumwidth}{12mm}
\setlength{\cftsubsecindent}{18mm}

\usepackage{fontspec}
\usepackage{xltxtra}
\defaultfontfeatures{Ligatures=TeX,Numbers=OldStyle}
\setromanfont{DejaVu Sans}
\setmonofont{DejaVu Sans Mono}[Scale={0.90}]
% Nota: La lista de fuentes disponibles se obtiene con fc-list

\usepackage[spanish]{babel}
\usepackage{fancyvrb}
\usepackage{a4wide}
\usepackage{minted}

% \usepackage{titletoc}
% % \contentsmargin{2.55em}
% % \dottedcontents{chapter}[0em]{}{32em}{1pc}
% \dottedcontents{section}[4.5em]{}{3.3em}{1pc}

\linespread{1.05}
\setlength{\parindent}{2em}
\raggedbottom

\usepackage[%
  colorlinks=true,
  urlcolor=blue,
  % pdftex,
  pdfauthor={José A. Alonso <jalonso@us.es>},%
  pdftitle={Piensa en Haskell y en Python},%
  pdfstartview=FitH,%
  bookmarks=false]{hyperref}

\setcounter{tocdepth}{1}
% \setcounter{secnumdepth}{4}

% \usepackage{tocstyle}
% \usetocstyle{KOMAlike}

% \usepackage{tocloft}
% \renewcommand\cftpartnumwidth{3cm}

\usepackage{minitoc}
% \setlength\cftparskip{-2pt}
% \setlength\cftbeforechapskip{0pt}

%%%%%%%%%%%%%%%%%%%%%%%%%%%%%%%%%%%%%%%%%%%%%%%%%%%%%%%%%%%%%%%%%%%%%%%%%%%%%%
%% § Cabeceras                                                              %%
%%%%%%%%%%%%%%%%%%%%%%%%%%%%%%%%%%%%%%%%%%%%%%%%%%%%%%%%%%%%%%%%%%%%%%%%%%%%%%

\usepackage{fancyhdr}

\addtolength{\headheight}{\baselineskip}

\pagestyle{fancy}

\cfoot{}

\fancyhead{}
\fancyhead[RE]{\mdseries\sffamily Ejercicios de programación con Python}
\fancyhead[LO]{\mdseries\sffamily \nouppercase{\leftmark}}
\fancyhead[LE,RO]{\mdseries\sffamily \thepage}

%%%%%%%%%%%%%%%%%%%%%%%%%%%%%%%%%%%%%%%%%%%%%%%%%%%%%%%%%%%%%%%%%%%%%%%%
%% § Definiciones
%%%%%%%%%%%%%%%%%%%%%%%%%%%%%%%%%%%%%%%%%%%%%%%%%%%%%%%%%%%%%%%%%%%%%%%%

\input definiciones
\def\mtctitle{Contenido}

%%%%%%%%%%%%%%%%%%%%%%%%%%%%%%%%%%%%%%%%%%%%%%%%%%%%%%%%%%%%%%%%%%%%%%%%
%% § Título
%%%%%%%%%%%%%%%%%%%%%%%%%%%%%%%%%%%%%%%%%%%%%%%%%%%%%%%%%%%%%%%%%%%%%%%%

\title{{\Huge Piensa en Haskell y en Python} \\
       {\large (Ejercicios de programación con Haskell y con Python)} }
\author{\href{http://www.cs.us.es/~jalonso}
             {\Large José A. Alonso Jiménez} }
\date{\vfill \hrule \vspace*{2mm}
  \begin{tabular}{l}
      \href{http://www.cs.us.es/glc}
           {Grupo de Lógica Computacional} \\
      \href{http://www.cs.us.es}
           {Dpto. de Ciencias de la Computación e Inteligencia Artificial} \\
      \href{http://www.us.es}
           {Universidad de Sevilla}  \\
      Sevilla, \today
  \end{tabular}\hfill\mbox{}}

%%%%%%%%%%%%%%%%%%%%%%%%%%%%%%%%%%%%%%%%%%%%%%%%%%%%%%%%%%%%%%%%%%%%%%%%
%% § Documento
%%%%%%%%%%%%%%%%%%%%%%%%%%%%%%%%%%%%%%%%%%%%%%%%%%%%%%%%%%%%%%%%%%%%%%%%

% \includeonly{recusion_sobre_numeros_naturales}

% \includexmp{licencia}

\begin{document}

\maketitle
\newpage

\noindent
Esta obra está bajo una licencia Reconocimiento--NoComercial--CompartirIgual
2.5 Spain de Creative Commons. 

\vspace*{4ex}

\begin{center}
\fbox{
\begin{minipage}{\textwidth}
\vspace*{1ex}
\noindent                     
{\bf Se permite:}
\begin{itemize}                         
\item copiar, distribuir y comunicar públicamente la obra
\item hacer obras derivadas
\end{itemize}
         
\noindent       
{\bf Bajo las condiciones siguientes:}

\begin{minipage}[t]{20mm}
  \vspace*{0mm}\hspace*{1em}
  \includegraphics[scale=0.35]{deed.png}
\end{minipage}%
\begin{minipage}[t]{14cm}
  \vspace*{0mm}
  {\bf Reconocimiento}. Debe reconocer los créditos de la  obra de la manera
  especificada por el autor.  
\end{minipage}
\\[1ex]

\begin{minipage}[t]{20mm}
  \vspace*{0mm}\hspace*{1em}
  \includegraphics[scale=0.35]{deed-eu.png}
\end{minipage}%
\begin{minipage}[t]{14cm}
  \vspace*{2mm}
  {\bf No comercial}. No puede utilizar esta obra para fines comerciales.
\end{minipage}
\\[1ex]

\begin{minipage}[t]{20mm}
  \vspace*{0mm}\hspace*{1em}
  \includegraphics[scale=0.35]{deed_002.png}
\end{minipage}%
\begin{minipage}[t]{14cm}
  \vspace*{-2mm}
  {\bf Compartir bajo la misma licencia}. Si altera o transforma esta obra, o
  genera una obra derivada, sólo puede distribuir la obra generada bajo una
  licencia idéntica a ésta. 
\end{minipage}

\begin{itemize}    
\item Al reutilizar o distribuir la obra, tiene que dejar bien claro los
  términos de la licencia de esta obra. 
\item Alguna de estas condiciones puede no aplicarse si se obtiene el permiso
 del titular de los derechos de autor.
\end{itemize}
\vspace*{1ex}
\end{minipage}}
\end{center}

\vspace*{4ex}

Esto es un resumen del texto legal (la licencia completa). Para ver una copia
de esta licencia, visite 
\href{http://creativecommons.org/licenses/by-nc-sa/2.5/es/}
     {\url{http://creativecommons.org/licenses/by-nc-sa/2.5/es/}}
o envie una carta a Creative Commons, 559 Nathan Abbott Way, Stanford,
California 94305, USA.

%%% Local Variables: 
%%% mode: latex
%%% TeX-master: t
%%% End: 

\newpage

\dominitoc
\tableofcontents
\clearpage

% \renewcommand{\chaptername}{Ejercicio}

\chapter*{Introducción}

Este libro es una introducción a la programación con Haskell y Python, a través de
la resolución de ejercicios publicados diariamente en el blog
\href{https://www.glc.us.es/~jalonso/exercitium}{Exercitium}
\footnote{https://www.glc.us.es/~jalonso/exercitium}.
Estos ejercicios están organizados siguiendo el orden de los
\href{https://jaalonso.github.io/materias/PFconHaskell/temas.html}
{Temas de programación funcional con Haskell}
\footnote{https://jaalonso.github.io/materias/PFconHaskell/temas.html}.
Las soluciones a los ejercicios están disponibles en dos repositorios de
GitHub, uno con las
\href{https://github.com/jaalonso/Exercitium}{soluciones en Haskell}
\footnote{https://github.com/jaalonso/Exercitium}
y otro con las
\href{https://github.com/jaalonso/Exercitium-Python}{soluciones en Python}
\footnote{https://github.com/jaalonso/Exercitium-Python}).
Ambos repositorios están estructurados como proyectos utilizando
\href{https://docs.haskellstack.org/en/stable/}{Stack}
\footnote{https://docs.haskellstack.org/en/stable} y
\href{https://python-poetry.org/}{Poetry}
\footnote{https://python-poetry.org}, respectivamente. Se han
escrito soluciones en Python con un estilo funcional similar al de
Haskell, y se han comprobado con
\href{http://mypy-lang.org}{mypy}
\footnote{http://mypy-lang.org}
que los tipos de las definiciones en Python son correctos. Además, se
han dado varias soluciones a cada ejercicio, verificando su
equivalencia (mediante
\href{https://hackage.haskell.org/package/QuickCheck}{QuickCheck}
\footnote{https://hackage.haskell.org/package/QuickCheck}
en Haskell e Hypothesis
\href{https://hypothesis.readthedocs.io/en/latest}{Hypothesis}
\footnote{https://hypothesis.readthedocs.io/en/latest}
en Python) y comparando su eficiencia.

El libro actualmente consta de cinco capítulos. Los tres primeros
capítulos tratan sobre cómo definir funciones usando composición,
comprensión y recursión. El cuarto capítulo introduce las funciones de
orden superior y el quinto muestra cómo definir y usar nuevos
tipos. Tenga en cuenta que algunas de las soluciones a los ejercicios
pueden no corresponder con el contenido del capítulo donde se
encuentran. En futuras versiones del libro, se ampliará el contenido
hasta completar el curso de
\href{https://jaalonso.github.io/materias/PFconHaskell}
{Programación funcional con Haskell}
\footnote{https://jaalonso.github.io/materias/PFconHaskell}.

\part{Introducción a la programación con Python}

% =====================================================================
\chapter{Definiciones elementales de funciones}
% =====================================================================

En este capítulo se plantean ejercicios con definiciones elementales (no
recursivas) de funciones. Se corresponden con los 4 primeros temas del
\href{https://jaalonso.github.io/materias/PFconHaskell/temas.html}
{Curso de programación funcional con Haskell}
\footnote{https://jaalonso.github.io/materias/PFconHaskell/temas.html}.

\minitoc

\section{Media aritmética de tres números}
\subsection*{En Haskell}
\haskell{Media_aritmetica_de_tres_numeros}
\subsection*{En Python}
\python{media_aritmetica_de_tres_numeros}

\section{Suma de monedas}
\subsection*{En Haskell}
\haskell{Suma_de_monedas}
\subsection*{En Python}
\python{suma_de_monedas}

\section{Volumen de la esfera}
\subsection*{En Haskell}
\haskell{Volumen_de_la_esfera}
\subsection*{En Python}
\python{volumen_de_la_esfera}

\section{Área de la corona circular}
\subsection*{En Haskell}
\haskell{Area_corona_circular}
\subsection*{En Python}
\python{area_corona_circular}

\section{Último dígito}
\subsection*{En Haskell}
\haskell{Ultimo_digito}
\subsection*{En Python}
\python{ultimo_digito}

\section{Máximo de tres números}
\subsection*{En Haskell}
\haskell{Maximo_de_tres_numeros}
\subsection*{En Python}
\python{maximo_de_tres_numeros}

\section{El primero al final}
\subsection*{En Haskell}
\haskell{El_primero_al_final}
\subsection*{En Python}
\python{el_primero_al_final}

\section{Los primeros al final}
\subsection*{En Haskell}
\haskell{Los_primeros_al_final}
\subsection*{En Python}
\python{los_primeros_al_final}

\section{Rango de una lista}
\subsection*{En Haskell}
\haskell{Rango_de_una_lista}
\subsection*{En Python}
\python{rango_de_una_lista}

\section{Reconocimiento de palindromos}
\subsection*{En Haskell}
\haskell{Reconocimiento_de_palindromos}
\subsection*{En Python}
\python{reconocimiento_de_palindromos}

\section{Interior de una lista}
\subsection*{En Haskell}
\haskell{Interior_de_una_lista}
\subsection*{En Python}
\python{interior_de_una_lista}

\section{Elementos finales}
\subsection*{En Haskell}
\haskell{Elementos_finales}
\subsection*{En Python}
\python{elementos_finales}

\section{Segmento de una lista}
\subsection*{En Haskell}
\haskell{Segmento_de_una_lista}
\subsection*{En Python}
\python{segmento_de_una_lista}

\section{Primeros y últimos elementos}
\subsection*{En Haskell}
\haskell{Primeros_y_ultimos_elementos}
\subsection*{En Python}
\python{primeros_y_ultimos_elementos}

\section{Elemento mediano}
\subsection*{En Haskell}
\haskell{Elemento_mediano}
\subsection*{En Python}
\python{elemento_mediano}

\section{Tres iguales}
\subsection*{En Haskell}
\haskell{Tres_iguales}
\subsection*{En Python}
\python{tres_iguales}

\section{Tres diferentes}
\subsection*{En Haskell}
\haskell{Tres_diferentes}
\subsection*{En Python}
\python{tres_diferentes}

\section{División segura}
\subsection*{En Haskell}
\haskell{Division_segura}
\subsection*{En Python}
\python{division_segura}

\section{Disyunción excluyente}
\subsection*{En Haskell}
\haskell{Disyuncion_excluyente}
\subsection*{En Python}
\python{disyuncion_excluyente}

\section{Mayor rectángulo}
\subsection*{En Haskell}
\haskell{Mayor_rectangulo}
\subsection*{En Python}
\python{mayor_rectangulo}

\section{Intercambio de componentes de un par}
\subsection*{En Haskell}
\haskell{Intercambio_de_componentes_de_un_par}
\subsection*{En Python}
\python{intercambio_de_componentes_de_un_par}

\section{Distancia entre dos puntos}
\subsection*{En Haskell}
\haskell{Distancia_entre_dos_puntos}
\subsection*{En Python}
\python{distancia_entre_dos_puntos}

\section{Permutación cíclica}
\subsection*{En Haskell}
\haskell{Permutacion_ciclica}
\subsection*{En Python}
\python{permutacion_ciclica}

\section{Mayor número con dos dígitos dados}
\subsection*{En Haskell}
\haskell{Mayor_numero_con_dos_digitos_dados}
\subsection*{En Python}
\python{mayor_numero_con_dos_digitos_dados}

\section{Número de raíces de la ecuación de segundo grado}
\subsection*{En Haskell}
\haskell{Numero_de_raices_de_la_ecuacion_de_segundo_grado}
\subsection*{En Python}
\python{numero_de_raices_de_la_ecuacion_de_segundo_grado}

\section{Raíces de la ecuación de segundo grado}
\subsection*{En Haskell}
\haskell{Raices_de_la_ecuacion_de_segundo_grado}
\subsection*{En Python}
\python{raices_de_la_ecuacion_de_segundo_grado}

\section{Fórmula de Herón para el área de un triángulo}
\subsection*{En Haskell}
\haskell{Formula_de_Heron_para_el_area_de_un_triangulo}
\subsection*{En Python}
\python{formula_de_Heron_para_el_area_de_un_triangulo}

\section{Intersección de intervalos cerrados}
\subsection*{En Haskell}
\haskell{Interseccion_de_intervalos_cerrados}
\subsection*{En Python}
\python{interseccion_de_intervalos_cerrados}

\section{Números racionales}
\subsection*{En Haskell}
\haskell{Numeros_racionales}
\subsection*{En Python}
\python{numeros_racionales}

% =====================================================================
\chapter{Definiciones por comprensión}
% =====================================================================

En este capítulo se presentan ejercicios con definiciones por
comprensión. Se corresponden con el
\href{https://jaalonso.github.io/materias/PFconHaskell/temas/tema-5.html}
{tema 5 del curso de programación funcional con Haskell}
\footnote{https://jaalonso.github.io/materias/PFconHaskell/temas/tema-5.html}.

\minitoc

\section{Reconocimiento de subconjunto}
\subsection*{En Haskell}
\haskell{Reconocimiento_de_subconjunto}
\subsection*{En Python}
\python{reconocimiento_de_subconjunto}

\section{Igualdad de conjuntos}
\subsection*{En Haskell}
\haskell{Igualdad_de_conjuntos}
\subsection*{En Python}
\python{igualdad_de_conjuntos}

\section{Unión conjuntista de listas}
\subsection*{En Haskell}
\haskell{Union_conjuntista_de_listas}
\subsection*{En Python}
\python{union_conjuntista_de_listas}

\section{Intersección conjuntista de listas}
\subsection*{En Haskell}
\haskell{Interseccion_conjuntista_de_listas}
\subsection*{En Python}
\python{interseccion_conjuntista_de_listas}

\section{Diferencia conjuntista de listas}
\subsection*{En Haskell}
\haskell{Diferencia_conjuntista_de_listas}
\subsection*{En Python}
\python{diferencia_conjuntista_de_listas}

\section{Divisores de un número}
\subsection*{En Haskell}
\haskell{Divisores_de_un_numero}
\subsection*{En Python}
\python{divisores_de_un_numero}

\section{Divisores primos}
\subsection*{En Haskell}
\haskell{Divisores_primos}
\subsection*{En Python}
\python{divisores_primos}

\section{Números libres de cuadrados}
\subsection*{En Haskell}
\haskell{Numeros_libres_de_cuadrados}
\subsection*{En Python}
\python{numeros_libres_de_cuadrados}

\section{Suma de los primeros números naturales}
\subsection*{En Haskell}
\haskell{Suma_de_los_primeros_numeros_naturales}
\subsection*{En Python}
\python{suma_de_los_primeros_numeros_naturales}

\section{Suma de los cuadrados de los primeros números naturales}
\subsection*{En Haskell}
\haskell{Suma_de_los_cuadrados_de_los_primeros_numeros_naturales}
\subsection*{En Python}
\python{suma_de_los_cuadrados_de_los_primeros_numeros_naturales}

\section{Suma de cuadrados menos cuadrado de la suma}
\subsection*{En Haskell}
\haskell{Suma_de_cuadrados_menos_cuadrado_de_la_suma}
\subsection*{En Python}
\python{suma_de_cuadrados_menos_cuadrado_de_la_suma}

\section{Triángulo aritmético}
\subsection*{En Haskell}
\haskell{Triangulo_aritmetico}
\subsection*{En Python}
\python{triangulo_aritmetico}

\section{Suma de divisores}
\subsection*{En Haskell}
\haskell{Suma_de_divisores}
\subsection*{En Python}
\python{suma_de_divisores}

\section{Números perfectos}
\subsection*{En Haskell}
\haskell{Numeros_perfectos}
\subsection*{En Python}
\python{numeros_perfectos}

\section{Números abundantes}
\subsection*{En Haskell}
\haskell{Numeros_abundantes}
\subsection*{En Python}
\python{numeros_abundantes}

\section{Números abundantes menores o iguales que n}
\subsection*{En Haskell}
\haskell{Numeros_abundantes_menores_o_iguales_que_n}
\subsection*{En Python}
\python{numeros_abundantes_menores_o_iguales_que_n}

\section{Todos los abundantes hasta n son pares}
\subsection*{En Haskell}
\haskell{Todos_los_abundantes_hasta_n_son_pares}
\subsection*{En Python}
\python{todos_los_abundantes_hasta_n_son_pares}

\section{Números abundantes impares}
\subsection*{En Haskell}
\haskell{Numeros_abundantes_impares}
\subsection*{En Python}
\python{numeros_abundantes_impares}

\section{Suma de múltiplos de 3 ó 5}
\subsection*{En Haskell}
\haskell{Suma_de_multiplos_de_3_o_5}
\subsection*{En Python}
\python{suma_de_multiplos_de_3_o_5}

\section{Puntos dentro del círculo}
\subsection*{En Haskell}
\haskell{Puntos_dentro_del_circulo}
\subsection*{En Python}
\python{puntos_dentro_del_circulo}

\section{Aproximación del número e}
\subsection*{En Haskell}
\haskell{Aproximacion_del_numero_e}
\subsection*{En Python}
\python{aproximacion_del_numero_e}

\section{Aproximación al límite de sen(x)/x cuando x tiende a cero}
\subsection*{En Haskell}
\haskell{Limite_del_seno}
\subsection*{En Python}
\python{limite_del_seno}

\section{Cálculo del número π mediante la fórmula de Leibniz}
\subsection*{En Haskell}
\haskell{Calculo_de_pi_mediante_la_formula_de_Leibniz}
\subsection*{En Python}
\python{calculo_de_pi_mediante_la_formula_de_Leibniz}

\section{Ternas pitagóricas}
\subsection*{En Haskell}
\haskell{Ternas_pitagoricas}
\subsection*{En Python}
\python{ternas_pitagoricas}

\section{Ternas pitagóricas con suma dada}
\subsection*{En Haskell}
\haskell{Ternas_pitagoricas_con_suma_dada}
\subsection*{En Python}
\python{ternas_pitagoricas_con_suma_dada}

\section{Producto escalar}
\subsection*{En Haskell}
\haskell{Producto_escalar}
\subsection*{En Python}
\python{producto_escalar}

\section{Representación densa de polinomios}
\subsection*{En Haskell}
\haskell{Representacion_densa_de_polinomios}
\subsection*{En Python}
\python{representacion_densa_de_polinomios}

\section{Base de datos de actividades.}
\subsection*{En Haskell}
\haskell{Base_de_dato_de_actividades}
\subsection*{En Python}
\python{base_de_dato_de_actividades}

% =====================================================================
\chapter{Definiciones por recursión}
% =====================================================================

En este capítulo se presentan ejercicios con definiciones por
comprensión. Se corresponden con el
\href{https://jaalonso.github.io/materias/PFconHaskell/temas/tema-6.html}
{tema 6 del curso de programación funcional con Haskell}
\footnote{https://jaalonso.github.io/materias/PFconHaskell/temas/tema-6.html}.

\minitoc

\section{Potencia entera}
\subsection*{En Haskell}
\haskell{Potencia_entera}
\subsection*{En Python}
\python{potencia_entera}

\section{Algoritmo de Euclides del mcd}
\subsection*{En Haskell}
\haskell{Algoritmo_de_Euclides_del_mcd}
\subsection*{En Python}
\python{algoritmo_de_Euclides_del_mcd}

\section{Dígitos de un número}
\subsection*{En Haskell}
\haskell{Digitos_de_un_numero}
\subsection*{En Python}
\python{digitos_de_un_numero}

\section{Suma de los digitos de un número}
\subsection*{En Haskell}
\haskell{Suma_de_los_digitos_de_un_numero}
\subsection*{En Python}
\python{suma_de_los_digitos_de_un_numero}

\section{Número a partir de sus dígitos}
\subsection*{En Haskell}
\haskell{Numero_a_partir_de_sus_digitos}
\subsection*{En Python}
\python{numero_a_partir_de_sus_digitos}

\section{Exponente de la mayor potencia de x que divide a y}
\subsection*{En Haskell}
\haskell{Exponente_mayor}
\subsection*{En Python}
\python{exponente_mayor}

\section{Producto cartesiano de dos conjuntos}
\subsection*{En Haskell}
\haskell{Producto_cartesiano_de_dos_conjuntos}
\subsection*{En Python}
\python{producto_cartesiano_de_dos_conjuntos}

\section{Subconjuntos de un conjunto}
\subsection*{En Haskell}
\haskell{Subconjuntos_de_un_conjunto}
\subsection*{En Python}
\python{subconjuntos_de_un_conjunto}

\section{El algoritmo de Luhn}
\subsection*{En Haskell}
\haskell{El_algoritmo_de_Luhn}
\subsection*{En Python}
\python{el_algoritmo_de_Luhn}

\section{Números de Lychrel}
\subsection*{En Haskell}
\haskell{Numeros_de_Lychrel}
\subsection*{En Python}
\python{numeros_de_Lychrel}

\section{Suma de los dígitos de una cadena}
\subsection*{En Haskell}
\haskell{Suma_de_digitos_de_cadena}
\subsection*{En Python}
\python{suma_de_digitos_de_cadena}

\section{Primera en mayúscula y restantes en minúscula}
\subsection*{En Haskell}
\haskell{Mayuscula_inicial}
\subsection*{En Python}
\python{mayuscula_inicial}

\section{Mayúsculas iniciales}
\subsection*{En Haskell}
\haskell{Mayusculas_iniciales}
\subsection*{En Python}
\python{mayusculas_iniciales}

\section{Posiciones de un carácter en una cadena}
\subsection*{En Haskell}
\haskell{Posiciones_de_un_caracter_en_una_cadena}
\subsection*{En Python}
\python{posiciones_de_un_caracter_en_una_cadena}

\section{Reconocimiento de subcadenas}
\subsection*{En Haskell}
\haskell{Reconocimiento_de_subcadenas}
\subsection*{En Python}
\python{reconocimiento_de_subcadenas}

% =====================================================================
\chapter{Funciones de orden superior}
% =====================================================================

En este capítulo se presentan ejercicios con definiciones por
comprensión. Se corresponden con el
\href{https://jaalonso.github.io/materias/PFconHaskell/temas/tema-7.html}
{tema 7 del curso de programación funcional con Haskell}
\footnote{https://jaalonso.github.io/materias/PFconHaskell/temas/tema-7.html}.

\minitoc

\section{Segmentos cuyos elementos cumplen una propiedad}
\subsection*{En Haskell}
\haskell{Segmentos_cuyos_elementos_cumple_una_propiedad}
\subsection*{En Python}
\python{segmentos_cuyos_elementos_cumple_una_propiedad}

\section{Elementos consecutivos relacionados}
\subsection*{En Haskell}
\haskell{Elementos_consecutivos_relacionados}
\subsection*{En Python}
\python{elementos_consecutivos_relacionados}

\section{Agrupación de elementos por posición}
\subsection*{En Haskell}
\haskell{Agrupacion_de_elementos_por_posicion}
\subsection*{En Python}
\python{agrupacion_de_elementos_por_posicion}

\section{Concatenación de una lista de listas}
\subsection*{En Haskell}
\haskell{Contenacion_de_una_lista_de_listas}
\subsection*{En Python}
\python{concatenacion_de_una_lista_de_listas}

\section{Aplica según propiedad}
\subsection*{En Haskell}
\haskell{Aplica_segun_propiedad}
\subsection*{En Python}
\python{aplica_segun_propiedad}

\section{Máximo de una lista}
\subsection*{En Haskell}
\haskell{Maximo_de_una_lista}
\subsection*{En Python}
\python{maximo_de_una_lista}

% =====================================================================
\chapter{Tipos definidos y tipos de datos algebraicos}
% =====================================================================

En este capítulo se presentan ejercicios con definiciones por
comprensión. Se corresponden con el
\href{https://jaalonso.github.io/materias/PFconHaskell/temas/tema-9.html}
{tema 9 del curso de programación funcional con Haskell}
\footnote{https://jaalonso.github.io/materias/PFconHaskell/temas/tema-9.html}.

\minitoc

\section{Movimientos en el plano}
\subsection*{En Haskell}
\haskell{Movimientos_en_el_plano}
\subsection*{En Python}
\python{movimientos_en_el_plano}

\section{El tipo de figuras geométricas}
\subsection*{En Haskell}
\haskell{El_tipo_de_figuras_geometricas}
\subsection*{En Python}
\python{el_tipo_de_figuras_geometricas}

\section{El tipo de los números naturales}
\subsection*{En Haskell}
\haskell{El_tipo_de_los_numeros_naturales}
\subsection*{En Python}
\python{el_tipo_de_los_numeros_naturales}

\section{El tipo de las listas}
\subsection*{En Haskell}
\haskell{El_tipo_de_las_listas}
\subsection*{En Python}
\python{el_tipo_de_las_listas}

\section{El tipo de los árboles binarios}
\subsection*{En Haskell}
\haskell{El_tipo_de_los_arboles_binarios}
\subsection*{En Python}
\python{el_tipo_de_los_arboles_binarios}

\section{El tipo de las fórmulas proposicionales}
\subsection{En Haskell}
\haskell{Tipo_de_formulas}
\subsection{En Python}
\python{tipo_de_formulas}

\section{El tipo de las fórmulas: Variables de una fórmula}
\subsection*{En Haskell}
\haskell{Variables_de_una_formula}
\subsection*{En Python}
\python{variables_de_una_formula}

\section{El tipo de las fórmulas: Valor de una fórmula}
\subsection*{En Haskell}
\haskell{Valor_de_una_formula}
\subsection*{En Python}
\python{valor_de_una_formula}

\section{El tipo de las fórmulas: Interpretaciones de una fórmula}
\subsection*{En Haskell}
\haskell{Interpretaciones_de_una_formula}
\subsection*{En Python}
\python{interpretaciones_de_una_formula}

\section{El tipo de las fórmulas: Reconocedor de tautologías}
\subsection*{En Haskell}
\haskell{Validez_de_una_formula}
\subsection*{En Python}
\python{validez_de_una_formula}

\section{El tipo de los árboles binarios con valores en las hojas}
\subsection{En Haskell}
\haskell{Arbol_binario_valores_en_hojas}
\subsection{En Python}
\python{arbol_binario_valores_en_hojas}

\section{Altura de un árbol binario}
\subsection*{En Haskell}
\haskell{Altura_de_un_arbol_binario}
\subsection*{En Python}
\python{altura_de_un_arbol_binario}

\section{Aplicación de una función a un árbol}
\subsection*{En Haskell}
\haskell{Aplicacion_de_una_funcion_a_un_arbol}
\subsection*{En Python}
\python{aplicacion_de_una_funcion_a_un_arbol}

\section{Árboles con la misma forma}
\subsection*{En Haskell}
\haskell{Arboles_con_la_misma_forma}
\subsection*{En Python}
\python{arboles_con_la_misma_forma}

\section{Árbol con las hojas en la profundidad dada}
\subsection*{En Haskell}
\haskell{Arbol_con_las_hojas_en_la_profundidad_dada}
\subsection*{En Python}
\python{arbol_con_las_hojas_en_la_profundidad_dada}

\section{El tipo de las expresiones aritméticas}
\subsection{En Haskell}
\haskell{Tipo_expresion_aritmetica}
\subsection{En Python}
\python{tipo_expresion_aritmetica}

\section{El tipo de las expresiones aritméticas: Valor de una expresión}
\subsection*{En Haskell}
\haskell{Valor_de_una_expresion_aritmetica}
\subsection*{En Python}
\python{valor_de_una_expresion_aritmetica}

\section{El tipo de las expresiones aritméticas: Valor de la resta}
\subsection*{En Haskell}
\haskell{Valor_de_la_resta}
\subsection*{En Python}
\python{valor_de_la_resta}

\section{El tipo de los árboles binarios con valores en nodos y hojas}
\subsection{En Haskell}
\haskell{Arboles_binarios}
\subsection{En Python}
\python{arboles_binarios}

\section{Número de hojas de un árbol binario}
\subsection*{En Haskell}
\haskell{Numero_de_hojas_de_un_arbol_binario}
\subsection*{En Python}
\python{numero_de_hojas_de_un_arbol_binario}

\section{Profundidad de un árbol binario}
\subsection{En Haskell}
\haskell{Profundidad_de_un_arbol_binario}
\subsection*{En Python}
\python{profundidad_de_un_arbol_binario}

\section{Recorrido de árboles binarios}
\subsection{En Haskell}
\haskell{Recorrido_de_arboles_binarios}
\subsection{En Python}
\python{recorrido_de_arboles_binarios}

\section{Imagen especular de un árbol binario}
\subsection{En Haskell}
\haskell{Imagen_especular_de_un_arbol_binario}
\subsection{En Python}
\python{imagen_especular_de_un_arbol_binario}

\section{Subárbol de profundidad dada}
\subsection{En Haskell}
\haskell{Subarbol_de_profundidad_dada}
\subsection{En Python}
\python{subarbol_de_profundidad_dada}

\section{Árbol de profundidad n con nodos iguales}
\subsection{En Haskell}
\haskell{Arbol_de_profundidad_n_con_nodos_iguales}
\subsection{En Python}
\python{arbol_de_profundidad_n_con_nodos_iguales}

\section{Suma de un árbol}
\subsection{En Haskell}
\haskell{Suma_de_un_arbol}
\subsection{En Python}
\python{suma_de_un_arbol}

\section{Rama izquierda de un árbol binario}
\subsection{En Haskell}
\haskell{Rama_izquierda_de_un_arbol_binario}
\subsection{En Python}
\python{rama_izquierda_de_un_arbol_binario}

\section{Árboles balanceados}
\subsection{En Haskell}
\haskell{Arboles_balanceados}
\subsection{En Python}
\python{arboles_balanceados}

\section{Árboles con bordes iguales}
\subsection{En Haskell}
\haskell{Arboles_con_bordes_iguales}
\subsection{En Python}
\python{arboles_con_bordes_iguales}

\section{Árboles con igual estructura}
\subsection{En Haskell}
\haskell{Arboles_con_igual_estructura}
\subsection{En Python}
\python{arboles_con_igual_estructura}

\section{Existencia de elementos del árbol que verifican una propiedad}
\subsection{En Haskell}
\haskell{Existencia_de_elemento_del_arbol_con_propiedad}
\subsection{En Python}
\python{existencia_de_elemento_del_arbol_con_propiedad}

\section{Elementos del nivel k de un árbol}
\subsection{En Haskell}
\haskell{Elementos_del_nivel_k_de_un_arbol}
\subsection{En Python}
\python{elementos_del_nivel_k_de_un_arbol}

\section{Árbol de factorización}
\subsection{En Haskell}
\haskell{Arbol_de_factorizacion}
\subsection{En Python}
\python{arbol_de_factorizacion}

\section{Valor de un árbol booleano}
\subsection{En Haskell}
\haskell{Valor_de_un_arbol_booleano}
\subsection{En Python}
\python{valor_de_un_arbol_booleano}

\section{Expresiones aritméticas básicas}
\subsection{En Haskell}
\haskell{Expresion_aritmetica_basica}
\subsection{En Python}
\python{expresion_aritmetica_basica}

\section{Valor de una expresión aritmética básica}
\subsection{En Haskell}
\haskell{Valor_de_una_expresion_aritmetica_basica}
\subsection{En Python}
\python{valor_de_una_expresion_aritmetica_basica}

\section{Aplicación de una función a una expresión aritmética}
\subsection{En Haskell}
\haskell{Aplicacion_de_una_funcion_a_una_expresion_aritmetica}
\subsection{En Python}
\python{aplicacion_de_una_funcion_a_una_expresion_aritmetica}

\section{Tipo de expresiones aritméticas con una variable}
\subsection{En Haskell}
\haskell{Expresion_aritmetica_con_una_variable}
\subsection{En Python}
\python{expresion_aritmetica_con_una_variable}

\section{Valor de una expresión aritmética con una variable}
\subsection{En Haskell}
\haskell{Valor_de_una_expresion_aritmetica_con_una_variable}
\subsection{En Python}
\python{valor_de_una_expresion_aritmetica_con_una_variable}

\section{Número de variables de una expresión aritmética}
\subsection{En Haskell}
\haskell{Numero_de_variables_de_una_expresion_aritmetica}
\subsection{En Python}
\python{numero_de_variables_de_una_expresion_aritmetica}

\section{Valor de una expresión aritmética con variables}
\subsection{En Haskell}
\haskell{Valor_de_una_expresion_aritmetica_con_variables}
\subsection{En Python}
\python{valor_de_una_expresion_aritmetica_con_variables}

\section{Número de sumas en una expresión aritmética}
\subsection{En Haskell}
\haskell{Numero_de_sumas_en_una_expresion_aritmetica}
\subsection{En Python}
\python{numero_de_sumas_en_una_expresion_aritmetica}

\section{Sustitución en una expresión aritmética}
\subsection{En Haskell}
\haskell{Sustitucion_en_una_expresion_aritmetica}
\subsection{En Python}
\python{sustitucion_en_una_expresion_aritmetica}

\section{Expresiones aritméticas reducibles}
\subsection{En Haskell}
\haskell{Expresiones_aritmeticas_reducibles}
\subsection{En Python}
\python{expresiones_aritmeticas_reducibles}

\section{Máximos valores de una expresión aritmética}
\subsection{En Haskell}
\haskell{Maximos_valores_de_una_expresion_aritmetica}
\subsection{En Python}
\python{maximos_valores_de_una_expresion_aritmetica}

\section{Valor de expresiones aritméticas generales}
\subsection{En Haskell}
\haskell{Valor_de_expresiones_aritmeticas_generales}
\subsection{En Python}
\python{valor_de_expresiones_aritmeticas_generales}

\section{Valor de una expresión vectorial}
\subsection{En Haskell}
\haskell{Valor_de_una_expresion_vectorial}
\subsection{En Python}
\python{valor_de_una_expresion_vectorial}

\part{Algorítmica}

\chapter{El tipo abstracto de datos de las pilas}

\minitoc

\section{El tipo abstracto de datos de las pilas}
\subsection{En Haskell}
\haskell{TAD/Pila}
\subsection{En Python}
\python{TAD/pila}

\section{El tipo de datos de las pilas mediante listas}
\subsection{En Haskell}
\haskell{TAD/PilaConListas}
\subsection{En Python}
\python{TAD/pilaConListas}

\section{El tipo de datos de las pilas con librerías}
\subsection{En Haskell}
\haskell{TAD/PilaConSucesiones}
\subsection{En Python}
\python{TAD/pilaConDeque}

\section{Transformación entre pilas y listas}
\subsection{En Haskell}
\haskell{Transformaciones_pilas_listas}
\subsection{En Python}
\python{transformaciones_pilas_listas}

\section{Filtrado de pilas según una propiedad}
\subsection{En Haskell}
\haskell{FiltraPila}
\subsection{En Python}
\python{filtraPila}

\section{Aplicación de una función a los elementos de una pila}
\subsection{En Haskell}
\haskell{MapPila}
\subsection{En Python}
\python{mapPila}

\section{Pertenencia a una pila}
\subsection{En Haskell}
\haskell{PertenecePila}
\subsection{En Python}
\python{pertenecePila}

\section{Inclusión de pilas}
\subsection{En Haskell}
\haskell{ContenidaPila}
\subsection{En Python}
\python{contenidaPila}

\section{Reconocimiento de prefijos de pilas}
\subsection{En Haskell}
\haskell{PrefijoPila}
\subsection{En Python}
\python{prefijoPila}

\section{Reconocimiento de subpilas}
\subsection{En Haskell}
\haskell{SubPila}
\subsection{En Python}
\python{subPila}

\section{Reconocimiento de ordenación de pilas}
\subsection{En Haskell}
\haskell{OrdenadaPila}
\subsection{En Python}
\python{ordenadaPila}

\section{Ordenación de pilas por inserción}
\subsection{En Haskell}
\haskell{OrdenaInserPila}
\subsection{En Python}
\python{ordenaInserPila}

\section{Eliminación de repeticiones en una pila}
\subsection{En Haskell}
\haskell{NubPila}
\subsection{En Python}
\python{nubPila}

\section{Máximo elemento de una pila}
\subsection{En Haskell}
\haskell{MaxPila}
\subsection{En Python}
\python{maxPila}

\chapter{El tipo abstracto de datos de las colas}

\minitoc

\section{El tipo abstracto de datos de las colas}
\subsection{En Haskell}
\haskell{TAD/Cola}
\subsection{En Python}
\python{TAD/cola}

\section{El tipo de datos de las colas mediante listas}
\subsection{En Haskell}
\haskell{TAD/ColaConListas}
\subsection{En Python}
\python{TAD/colaConListas}

\section{El tipo de datos de las colas mediante dos listas}
\subsection{En Haskell}
\haskell{TAD/ColaConDosListas}
\subsection{En Python}
\python{TAD/colaConDosListas}

\section{El tipo de datos de las colas mediante sucesiones}
\subsection{En Haskell}
\haskell{TAD/ColaConSucesiones}
\subsection{En Python}
\python{TAD/colaConDeque}

\section{Transformaciones entre colas y listas}
\subsection{En Haskell}
\haskell{Transformaciones_colas_listas}
\subsection{En Python}
\python{transformaciones_colas_listas}

\section{Último elemento de una cola}
\subsection{En Haskell}
\haskell{UltimoCola}
\subsection{En Python}
\python{ultimoCola}

\section{Longitud de una cola}
\subsection{En Haskell}
\haskell{LongitudCola}
\subsection{En Python}
\python{longitudCola}

\section{Todos los elementos de la cola verifican una propiedad}
\subsection{En Haskell}
\haskell{TodosVerifican}
\subsection{En Python}
\python{todosVerifican}

\section{Algún elemento de la verifica una propiedad}
\subsection{En Haskell}
\haskell{AlgunoVerifica}
\subsection{En Python}
\python{algunoVerifica}

\section{Extensión de colas}
\subsection{En Haskell}
\haskell{ExtiendeCola}
\subsection{En Python}
\python{extiendeCola}

\section{Intercalado de dos colas}
\subsection{En Haskell}
\haskell{IntercalaColas}
\subsection{En Python}
\python{intercalaColas}

\section{Agrupación de colas}
\subsection{En Haskell}
\haskell{AgrupaColas}
\subsection{En Python}
\python{agrupaColas}

\section{Pertenencia a una cola}
\subsection{En Haskell}
\haskell{PerteneceCola}
\subsection{En Python}
\python{perteneceCola}

\section{Inclusión de colas}
\subsection{En Haskell}
\haskell{ContenidaCola}
\subsection{En Python}
\python{contenidaCola}

\section{Reconocimiento de prefijos de colas}
\subsection{En Haskell}
\haskell{PrefijoCola}
\subsection{En Python}
\python{prefijoCola}

\section{Reconocimiento de subcolas}
\subsection{En Haskell}
\haskell{SubCola}
\subsection{En Python}
\python{subCola}

\section{Reconocimiento de ordenación de colas}
\subsection{En Haskell}
\haskell{OrdenadaCola}
\subsection{En Python}
\python{ordenadaCola}

)\section{Máximo elemento de una cola}
\subsection{En Haskell}
\haskell{MaxCola}
\subsection{En Python}
\python{maxCola}

\chapter{El tipo abstracto de datos de los conjuntos}

\minitoc

\section{El tipo abstracto de datos de los conjuntos}
\subsection{En Haskell}
\haskell{TAD/Conjunto}
\subsection{En Python}
\python{TAD/conjunto}

\section{El tipo de datos de los conjuntos mediante listas no ordenadas con duplicados}
\subsection{En Haskell}
\haskell{TAD/ConjuntoConListasNoOrdenadasConDuplicados}
\subsection{En Python}
\python{TAD/conjuntoConListasNoOrdenadasConDuplicados}

\section{El tipo de datos de los conjuntos mediante listas no ordenadas sin duplicados}
\subsection{En Haskell}
\haskell{TAD/ConjuntoConListasNoOrdenadasSinDuplicados}
\subsection{En Python}
\python{TAD/conjuntoConListasNoOrdenadasSinDuplicados}

\section{El tipo de datos de los conjuntos mediante listas ordenadas sin duplicados}
\subsection{En Haskell}
\haskell{TAD/ConjuntoConListasOrdenadasSinDuplicados}
\subsection{En Python}
\python{TAD/conjuntoConListasOrdenadasSinDuplicados}

\section{El tipo de datos de los conjuntos mediante librería}
\subsection{En Haskell}
\haskell{TAD/ConjuntoConLibreria}
\subsection{En Python}
\python{TAD/conjuntoConLibreria}

\section{Transformaciones entre conjuntos y listas}
\subsection{En Haskell}
\haskell{TAD_Transformaciones_conjuntos_listas}
\subsection{En Python}
\python{TAD_Transformaciones_conjuntos_listas}

\section{Reconocimiento de subconjunto}
\subsection{En Haskell}
\haskell{TAD_subconjunto}
\subsection{En Python}
\python{TAD_subconjunto}

\section{Reconocimiento de subconjunto propio}
\subsection{En Haskell}
\haskell{TAD_subconjuntoPropio}
\subsection{En Python}
\python{TAD_subconjuntoPropio}

\section{Conjunto unitario}
\subsection{En Haskell}
\haskell{TAD_Conjunto_unitario}
\subsection{En Python}
\python{TAD_Conjunto_unitario}

\section{Número de elementos de un conjunto}
\subsection{En Haskell}
\haskell{TAD_Numero_de_elementos_de_un_conjunto}
\subsection{En Python}
\python{TAD_Numero_de_elementos_de_un_conjunto}

\section{Unión de dos conjuntos}
\subsection{En Haskell}
\haskell{TAD_Union_de_dos_conjuntos}
\subsection{En Python}
\python{TAD_Union_de_dos_conjuntos}

\section{Unión de varios conjuntos}
\subsection{En Haskell}
\haskell{TAD_Union_de_varios_conjuntos}
\subsection{En Python}
\python{TAD_Union_de_varios_conjuntos}

\section{Intersección de dos conjuntos}
\subsection{En Haskell}
\haskell{TAD_Interseccion_de_dos_conjuntos}
\subsection{En Python}
\python{TAD_Interseccion_de_dos_conjuntos}

\section{Intersección de varios conjuntos}
\subsection{En Haskell}
\haskell{TAD_Interseccion_de_varios_conjuntos}
\subsection{En Python}
\python{TAD_Interseccion_de_varios_conjuntos}

\section{Conjuntos disjuntos}
\subsection{En Haskell}
\haskell{TAD_Conjuntos_disjuntos}
\subsection{En Python}
\python{TAD_Conjuntos_disjuntos}

\section{Diferencia de conjuntos}
\subsection{En Haskell}
\haskell{TAD_Diferencia_de_conjuntos}
\subsection{En Python}
\python{TAD_Diferencia_de_conjuntos}

\section{Diferencia simétrica}
\subsection{En Haskell}
\haskell{TAD_Diferencia_simetrica}
\subsection{En Python}
\python{TAD_Diferencia_simetrica}

\section{Subconjunto determinado por una propiedad}
\subsection{En Haskell}
\haskell{TAD_Subconjunto_por_propiedad}
\subsection{En Python}
\python{TAD_Subconjunto_por_propiedad}

\section{Partición de un conjunto según una propiedad}
\subsection{En Haskell}
\haskell{TAD_Particion_por_una_propiedad}
\subsection{En Python}
\python{TAD_Particion_por_una_propiedad}

\section{Partición según un número}
\subsection{En Haskell}
\haskell{TAD_Particion_segun_un_numero}
\subsection{En Python}
\python{TAD_Particion_segun_un_numero}

\section{Aplicación de una función a los elementos de un conjunto}
\subsection{En Haskell}
\haskell{TAD_mapC}
\subsection{En Python}
\python{TAD_mapC}

\section{Todos los elementos verifican una propiedad}
\subsection{En Haskell}
\haskell{TAD_TodosVerificanConj}
\subsection{En Python}
\python{TAD_TodosVerificanConj}

\section{Algunos elementos verifican una propiedad}
\subsection{En Haskell}
\haskell{TAD_AlgunosVerificanConj}
\subsection{En Python}
\python{TAD_AlgunosVerificanConj}

\section{Producto cartesiano}
\subsection{En Haskell}
\haskell{TAD_Producto_cartesiano}
\subsection{En Python}
\python{TAD_Producto_cartesiano}

\chapter{Relaciones binarias}

\minitoc

\section{El tipo de las relaciones binarias}
\subsection{En Haskell}
\haskell{Relaciones_binarias}
\subsection{En Python}
\python{Relaciones_binarias}

\section{Universo y grafo de una relación binaria}
\subsection{En Haskell}
\haskell{Universo_y_grafo_de_una_relacion_binaria}
\subsection{En Python}
\python{Universo_y_grafo_de_una_relacion_binaria}

\section{Relaciones reflexivas}
\subsection{En Haskell}
\haskell{Relaciones_reflexivas}
\subsection{En Python}
\python{Relaciones_reflexivas}

\section{Relaciones simétricas}
\subsection{En Haskell}
\haskell{Relaciones_simetricas}
\subsection{En Python}
\python{Relaciones_simetricas}

\section{Composición de relaciones binarias}
\subsection{En Haskell}
\haskell{Composicion_de_relaciones_binarias_v2}
\subsection{En Python}
\python{Composicion_de_relaciones_binarias_v2}

\section{Reconocimiento de subconjunto}
\subsection{En Haskell}
\haskell{Reconocimiento_de_subconjunto}
\subsection{En Python}
\python{Reconocimiento_de_subconjunto}

\section{Relaciones transitivas}
\subsection{En Haskell}
\haskell{Relaciones_transitivas}
\subsection{En Python}
\python{Relaciones_transitivas}

\section{Relaciones irreflexivas}
\subsection{En Haskell}
\haskell{Relaciones_irreflexivas}
\subsection{En Python}
\python{Relaciones_irreflexivas}

\section{Relaciones antisimétricas}
\subsection{En Haskell}
\haskell{Relaciones_antisimetricas}
\subsection{En Python}
\python{Relaciones_antisimetricas}

\section{Relaciones totales}
\subsection{En Haskell}
\haskell{Relaciones_totales}
\subsection{En Python}
\python{Relaciones_totales}

\section{Clausura reflexiva}
\subsection{En Haskell}
\haskell{Clausura_reflexiva}
\subsection{En Python}
\python{Clausura_reflexiva}

\section{Clausura simétrica}
\subsection{En Haskell}
\haskell{Clausura_simetrica}
\subsection{En Python}
\python{Clausura_simetrica}

\section{Clausura transitiva}
\subsection{En Haskell}
\haskell{Clausura_transitiva}
\subsection{En Python}
\python{Clausura_transitiva}

\chapter{El tipo abstracto de datos de los polinomios}

\minitoc

\section{El tipo abstracto de datos de los polinomios}
\subsection{En Haskell}
\haskell{TAD/Polinomio}
\subsection{En Python}
\python{TAD/Polinomio}

\section{El TAD de los polinomios mediante tipos algebraicos}
\subsection{En Haskell}
\haskell{TAD/PolRepTDA}

\section{El TAD de los polinomios mediante listas densas}
\subsection{En Haskell}
\haskell{TAD/PolRepDensa}
\subsection{En Python}
\python{TAD/PolRepDensa}

\section{El TAD de los polinomios mediante listas dispersas}
\subsection{En Haskell}
\haskell{TAD/PolRepDispersa}
\subsection{En Python}
\python{TAD/PolRepDispersa}

\section{Transformaciones entre las representaciones dispersa y densa}
\subsection{En Haskell}
\haskell{Pol_Transformaciones_dispersa_y_densa}
\subsection{En Python}
\python{Pol_Transformaciones_dispersa_y_densa}

\section{Transformaciones entre polinomios y listas dispersas}
\subsection{En Haskell}
\haskell{Pol_Transformaciones_polinomios_dispersas}
\subsection{En Python}
\python{Pol_Transformaciones_polinomios_dispersas}

\section{Coeficiente del término de grado k}
\subsection{En Haskell}
\haskell{Pol_Coeficiente}
\subsection{En Python}
\python{Pol_Coeficiente}

\section{Transformaciones entre polinomios y listas densas}
\subsection{En Haskell}
\haskell{Pol_Transformaciones_polinomios_densas}
\subsection{En Python}
\python{Pol_Transformaciones_polinomios_densas}

\part*{Apéndices}
\appendix

% A. Resumen de funciones Haskell
% % resumen_Haskell.tex
% Resumen de funciones habituales de Haskell.
% José A. Alonso Jiménez <jalonso@us,es>
% Sevilla, 13 de Noviembre de 2007
% ============================================================================

%%%%%%%%%%%%%%%%%%%%%%%%%%%%%%%%%%%%%%%%%%%%%%%%%%%%%%%%%%%%%%%%%%%%%%%%%%%%%%
%% § Paquetes adicionales                                                   %%
%%%%%%%%%%%%%%%%%%%%%%%%%%%%%%%%%%%%%%%%%%%%%%%%%%%%%%%%%%%%%%%%%%%%%%%%%%%%%%

\newcommand{\verba}[1]{%
  \fbox{\textcolor{blue}{\ \texttt{#1}\phantom{I}}}}

%%%%%%%%%%%%%%%%%%%%%%%%%%%%%%%%%%%%%%%%%%%%%%%%%%%%%%%%%%%%%%%%%%%%%%%%%%%%%%
%% § Documento                                                              %%
%%%%%%%%%%%%%%%%%%%%%%%%%%%%%%%%%%%%%%%%%%%%%%%%%%%%%%%%%%%%%%%%%%%%%%%%%%%%%%

\section{Resumen de funciones predefinidas de Haskell}

\begin{enumerate*}
\item \verba{x + y} es la suma de \verb|x| e \verb|y|.
\item \verba{x - y} es la resta de \verb|x| e \verb|y|.
\item \verba{x / y} es el cociente de \verb|x| entre \verb|y|.
\item \verba{x \^\ y} es \verb|x| elevado a \verb|y|.
\item \verba{x == y} se verifica si \verb|x| es igual a \verb|y|.
\item \verba{x /= y} se verifica si \verb|x| es distinto de \verb|y|.
\item \verba{x <\ y} se verifica si \verb|x| es menor que \verb|y|.
\item \verba{x <= y} se verifica si \verb|x| es menor o igual que \verb|y|.
\item \verba{x >\ y} se verifica si \verb|x| es mayor que \verb|y|.
\item \verba{x >= y} se verifica si \verb|x| es mayor o igual que \verb|y|.
\item \verba{x \&\& y} es la conjunción de \verb|x| e \verb|y|.
\item \verba{x || y} es la disyunción de \verb|x| e \verb|y|.
\item \verba{x:ys} es la lista obtenida añadiendo \verb|x| al principio de
  \verb|ys|.
\item \verba{xs ++ ys} es la concatenación de \verb|xs| e \verb|ys|.
\item \verba{xs !! n} es el elemento \verb|n|--ésimo de \verb|xs|.
\item \verba{f . g} es la composición de \verb|f| y \verb|g|.
\item \verba{abs x} es el valor absoluto de \verb|x|.
\item \verba{and xs} es la conjunción de la lista de booleanos \verb|xs|.
\item \verba{ceiling x} es el menor entero no menor que \verb|x|.
\item \verba{chr n} es el carácter cuyo código ASCII es \verb|n|.
\item \verba{concat xss} es la concatenación de la lista de listas \verb|xss|.
\item \verba{const x y} es \verb|x|.
\item \verba{curry f} es la versión curryficada de la función \verb|f|.
\item \verba{div x y} es la división entera de \verb|x| entre \verb|y|.
\item \verba{drop n xs} borra los \verb|n| primeros elementos de \verb|xs|.
\item \verba{dropWhile p xs} borra el mayor prefijo de \verb|xs| cuyos
  elementos satisfacen el predicado \verb|p|.
\item \verba{elem x ys} se verifica si \verb|x| pertenece a \verb|ys|.
\item \verba{even x} se verifica si \verb|x| es par.
\item \verba{filter p xs} es la lista de elementos de la lista \verb|xs| que
  verifican el predicado \verb|p|.
\item \verba{flip f x y} es \verb|f y x|.
\item \verba{floor x} es el mayor entero no mayor que \verb|x|.
\item \verba{foldl f e xs} pliega \verb|xs| de izquierda a derecha
  usando el operador f y el valor inicial \verb|e|.
\item \verba{foldr f e xs} pliega \verb|xs| de derecha a izquierda
  usando el operador f y el valor inicial \verb|e|.
\item \verba{fromIntegral x} transforma el número entero \verb|x| al tipo
  numérico correspondiente.
\item \verba{fst p} es el primer elemento del par \verb|p|.
\item \verba{gcd x y} es el máximo común divisor de de \verb|x| e \verb|y|.
\item \verba{head xs} es el primer elemento de la lista \verb|xs|.
\item \verba{init xs} es la lista obtenida eliminando el último elemento de
  \verb|xs|.
\item \verba{isSpace x}    se verifica si \verb|x| es un espacio.
\item \verba{isUpper x}    se verifica si \verb|x| está en mayúscula.
\item \verba{isLower x}    se verifica si \verb|x| está en minúscula.
\item \verba{isAlpha x}    se verifica si \verb|x| es un carácter alfabético.
\item \verba{isDigit x}    se verifica si \verb|x| es un dígito.
\item \verba{isAlphaNum x} se verifica si \verb|x| es un carácter alfanumérico.
\item \verba{iterate f x} es la lista \verb|[x, f(x), f(f(x)), ...]|.
\item \verba{last xs} es el último elemento de la lista \verb|xs|.
\item \verba{length xs} es el número de elementos de la lista \verb|xs|.
\item \verba{map f xs} es la lista obtenida aplicado \verb|f| a cada elemento
  de \verb|xs|.
\item \verba{max x y} es el máximo de \verb|x| e \verb|y|.
\item \verba{maximum xs} es el máximo elemento de la lista \verb|xs|.
\item \verba{min x y} es el mínimo de \verb|x| e \verb|y|.
\item \verba{minimum xs} es el mínimo elemento de la lista \verb|xs|.
\item \verba{mod x y} es el resto de \verb|x| entre \verb|y|.
\item \verba{not x} es la negación lógica del booleano \verb|x|.
\item \verba{noElem x ys} se verifica si \verb|x| no pertenece a \verb|ys|.
\item \verba{null xs} se verifica si \verb|xs| es la lista vacía.
\item \verba{odd x} se verifica si \verb|x| es impar.
\item \verba{or xs} es la disyunción de la lista de booleanos \verb|xs|.
\item \verba{ord c} es el código ASCII del carácter \verb|c|.
\item \verba{product xs} es el producto de la lista de números \verb|xs|.
\item \verba{rem x y} es el resto de \verb|x| entre \verb|y|.
\item \verba{repeat x} es la lista infinita \verb|[x, x, x, ...]|.
\item \verba{replicate n x} es la lista formada por \verb|n| veces el elemento
  \verb|x|.
\item \verba{reverse xs} es la inversa de la lista \verb|xs|.
\item \verba{round x} es el redondeo de \verb|x| al entero más cercano.
\item \verba{scanr f e xs} es la lista de los resultados de plegar \verb|xs|
  por la derecha con \verb|f| y \verb|e|.
\item \verba{show x} es la represantación de \verb|x| como cadena.
\item \verba{signum x} es 1 si \verb|x| es positivo, 0 si \verb|x| es cero y -1
  si \verb|x| es negativo.
\item \verba{snd p} es el segundo elemento del par \verb|p|.
\item \verba{splitAt n xs} es \verb|(take n xs, drop n xs)|.
\item \verba{sqrt x} es la raíz cuadrada de \verb|x|.
\item \verba{sum xs} es la suma de la lista numérica \verb|xs|.
\item \verba{tail xs} es la lista obtenida eliminando el primer elemento de
  \verb|xs|.
\item \verba{take n xs} es la lista de los \verb|n| primeros elementos de
  \verb|xs|.
\item \verba{takeWhile p xs} es el mayor prefijo de \verb|xs| cuyos elementos
  satisfacen el predicado \verb|p|.
\item \verba{uncurry f} es la versión cartesiana de la función \verb|f|.
\item \verba{until p f x} aplica \verb|f| a \verb|x| hasta que se verifique
  \verb|p|.
\item \verba{zip xs ys} es la lista de pares formado por los correspondientes
  elementos de \verb|xs| e \verb|ys|.
\item \verba{zipWith f xs ys} se obtiene aplicando \verb|f| a los
  correspondientes elementos de \verb|xs| e \verb|ys|.
\end{enumerate*}


% B. Método de Pólya para la resolución de problemas
\chapter{Método de Pólya para la resolución de problemas}

\section{Método de Pólya para la resolución de problemas matemáticos}

\noindent Para resolver un problema se necesita:
\subsubsection*{Paso 1: Entender el problema}
\begin{itemize}
\item ¿Cuál es la incógnita?, ¿Cuáles son los datos?
\item ¿Cuál es la condición? ¿Es la condición suficiente para determinar la
  incógnita? ¿Es insuficiente? ¿Redundante? ¿Contradictoria?
\end{itemize}

\subsubsection*{Paso 2: Configurar un plan}
\begin{itemize}
\item ¿Te has encontrado con un problema semejante? ¿O has visto el mismo
  problema planteado en forma ligeramente diferente?
\item ¿Conoces algún problema relacionado con éste? ¿Conoces algún teorema que
  te pueda ser útil? Mira atentamente la incógnita y trata de recordar un
  problema que sea familiar y que tenga la misma incógnita o una incógnita
  similar.
\item He aquí un problema relacionado al tuyo y que ya has resuelto ya. ¿Puedes
  utilizarlo? ¿Puedes utilizar su resultado? ¿Puedes emplear su método? ¿Te
  hace falta introducir algún elemento auxiliar a fin de poder utilizarlo?
\item ¿Puedes enunciar al problema de otra forma? ¿Puedes plantearlo en forma
  diferente nuevamente? Recurre a las definiciones.
\item Si no puedes resolver el problema propuesto, trata de resolver primero
  algún problema similar. ¿Puedes imaginarte un problema análogo un tanto más
  accesible? ¿Un problema más general? ¿Un problema más particular? ¿Un problema
  análogo? ¿Puede resolver una parte del problema? Considera sólo una parte de
  la condición; descarta la otra parte; ¿en qué medida la incógnita queda ahora
  determinada? ¿En qué forma puede variar? ¿Puedes deducir algún
  elemento útil de los datos? ¿Puedes pensar en algunos otros datos apropiados
  para determinar la incógnita? ¿Puedes cambiar la incógnita? ¿Puedes cambiar la
  incógnita o los datos, o ambos si es necesario, de tal forma que estén más
  cercanos entre sí?
\item ¿Has empleado todos los datos? ¿Has empleado toda la condición? ¿Has
  considerado todas las nociones esenciales concernientes al problema?
\end{itemize}

\subsubsection*{Paso 3: Ejecutar el plan}
\begin{itemize}
\item Al ejercutar tu plan de la solución, comprueba cada uno de los pasos
\item ¿Puedes ver claramente que el paso es correcto? ¿Puedes demostrarlo?
\end{itemize}

\subsubsection*{Paso 4: Examinar la solución obtenida}
\begin{itemize}
\item ¿Puedes verificar el resultado? ¿Puedes el razonamiento?
\item ¿Puedes obtener el resultado en forma diferente? ¿Puedes verlo de golpe?
  ¿Puedes emplear el resultado o el método en algún otro problema?
\end{itemize}

\noindent
\textit{G. Polya ``Cómo plantear y resolver problemas'' (Ed. Trillas, 1978)
  p. 19}

\section{Método de Pólya para resolver problemas de programación}

\noindent Para resolver un problema se necesita:
\subsubsection*{Paso 1: Entender el problema}
\begin{itemize}
\item ¿Cuáles son las \emph{argumentos}? ¿Cuál es el \emph{resultado}? ¿Cuál es
  \emph{nombre} de la función? ¿Cuál es su \emph{tipo}?
\item ¿Cuál es la \emph{especificación} del problema? ¿Puede satisfacerse la
  especificación? ¿Es insuficiente? ¿Redundante? ¿Contradictoria? ¿Qué
  restricciones se suponen sobre los argumentos y el resultado?
\item ¿Puedes descomponer el problema en partes? Puede ser útil dibujar
  diagramas con ejemplos de argumentos y resultados.
\end{itemize}

\subsubsection*{Paso 2: Diseñar el programa}
\begin{itemize}
\item ¿Te has encontrado con un problema semejante? ¿O has visto el mismo
  problema planteado en forma ligeramente diferente?
\item ¿Conoces algún problema \emph{relacionado} con éste? ¿Conoces alguna
  función que te pueda ser útil? Mira atentamente el tipo y trata de recordar un
  problema que sea familiar y que tenga el mismo tipo o un tipo similar.
\item ¿Conoces algún problema familiar con una \emph{especificación} similar?
\item He aquí un problema \emph{relacionado} al tuyo y que ya has
  resuelto. ¿Puedes utilizarlo? ¿Puedes utilizar su resultado? ¿Puedes emplear
  su método? ¿Te hace falta introducir alguna función auxiliar a fin de poder
  utilizarlo?
\item Si no puedes resolver el problema propuesto, trata de resolver primero
  algún problema similar. ¿Puedes imaginarte un problema análogo un
  tanto más \emph{accesible}? ¿Un problema más \emph{general}? ¿Un problema más
  \emph{particular}? ¿Un problema \emph{análogo}?
\item ¿Puede resolver una \emph{parte} del problema? ¿Puedes deducir algún
  elemento útil de los datos? ¿Puedes pensar en algunos otros datos apropiados
  para determinar la incógnita? ¿Puedes cambiar la incógnita? ¿Puedes cambiar la
  incógnita o los datos, o ambos si es necesario, de tal forma que estén más
  cercanos entre sí?
\item ¿Has empleado todos los datos? ¿Has empleado todas las restricciones
  sobre los datos? ¿Has considerado todas los requisitos de la especificación?
\end{itemize}

\subsubsection*{Paso 3: Escribir el programa}
\begin{itemize}
\item Al escribir el programa, comprueba cada uno de los pasos y funciones
  auxiliares.
\item ¿Puedes ver claramente que cada paso o función auxiliar es correcta?
\item Puedes escribir el programa en \emph{etapas}. Piensas en los diferentes
  \emph{casos} en los que se divide el problema; en particular, piensas en los
  diferentes casos para los datos. Puedes pensar en el cálculo de los casos
  independientemente y \emph{unirlos} para obtener el resultado final
\item Puedes pensar en la solución del problema descomponiéndolo en problemas
  con datos más simples y uniendo las soluciones parciales para obtener la
  solución del problema; esto es, por \emph{recursión}.
\item En su diseño se puede usar problemas más generales o más
  particulares. Escribe las soluciones de estos problemas; ellas puede servir
  como guía para la solución del problema original, o se pueden usar en su
  solución.
\item ¿Puedes apoyarte en otros problemas que has resuelto? ¿Pueden usarse?
  ¿Pueden modificarse? ¿Pueden guiar la solución del problema original?
\end{itemize}

\subsubsection*{Paso 4: Examinar la solución obtenida}
\begin{itemize}
\item ¿Puedes comprobar el funcionamiento del programa sobre una colección de
  argumentos?
\item ¿Puedes comprobar propiedades del programa?
\item ¿Puedes escribir el programa en una forma diferente?
\item ¿Puedes emplear el programa o el método en algún otro programa?
\end{itemize}

\noindent
Simon Thompson
\href{http://www.cs.kent.ac.uk/people/staff/sjt/Haskell_craft/HowToProgIt.html}
     {\emph{How to program it}},
basado en G. Polya \emph{Cómo plantear y resolver problemas}.


%%%%%%%%%%%%%%%%%%%%%%%%%%%%%%%%%%%%%%%%%%%%%%%%%%%%%%%%%%%%%%%%%%%%%%%%%%%%%%%
%%  Bibliografía                                                            %%
%%%%%%%%%%%%%%%%%%%%%%%%%%%%%%%%%%%%%%%%%%%%%%%%%%%%%%%%%%%%%%%%%%%%%%%%%%%%%%%

\nocite{Allen-16a}
\nocite{Alonso-12a}
\nocite{Alonso-19a}
\nocite{Alonso-21a}
\nocite{Alonso-22a}
\nocite{Bird-10}
\nocite{Bird-14a}
\nocite{Bird-99a}
\nocite{Bird-Gibbons-20a}
\nocite{Casamayou-12a}
\nocite{Downey-02a}
\nocite{Goodrich-13data}
\nocite{Guttag-16introduction}
\nocite{Hall-10a}
\nocite{Hetland-11apython}
\nocite{Hudak-12a}
\nocite{Hunt-19a}
\nocite{Hunt-19b}
\nocite{Hutton-16a}
\nocite{Kurt-18a}
\nocite{Lipovača}
\nocite{Lott-18a}
\nocite{OSullivan-08a}
\nocite{Okasaki-19a}
\nocite{Padmanabhan-17a}
\nocite{Polya-65a}
\nocite{Rabhi-99a}
\nocite{Rubio-17a}
\nocite{Ruiz-04}
\nocite{Saha-15a}
\nocite{Sajanikar-17a}
\nocite{Sannella-22a}
\nocite{Serrano-14a}
\nocite{Shukla-14a}
\nocite{Stephenson-15a}
\nocite{Thompson-11a}
\nocite{vanHattem-22a}

\addcontentsline{toc}{chapter}{Bibliografía}
\bibliographystyle{abbrv}
\bibliography{Piensa_en_Haskell_y_en_Python}

\end{document}

%%% Local Variables:
%%% mode: latex
%%% TeX-master: t
%%% End:
